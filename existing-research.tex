\chapter{Existing Research}
In \cite{nurse2015investigating}, the idea of using the information available in an email header was mooted, turning the previously standard threat of malware and phishing contained in received e-mails on its head, and instead presenting the threat in outgoing emails, and the personally identifying information (PII) contained therein.  Many emails leaked information about employers, e-mail services and applications used, and IP address.  Initial examination of a variety of e-mail headers found within my own inbox also revealed a plethora of information, including phone carriers, preferred languages, and system usernames.  It is conceivable therefore, that it is possible to automate at least part of this, and present the information that can be extracted, in a white-hat tool to allow people to audit the information that they are revealing.  The obvious malicious use-case involves using such information as part of a spear-phishing exercise.

An alternative vulnerability presents itself in the information about systems that may be revealed.  Many email clients embed identifying information, and there are multiple databases available to allow specific threats to be identified.  This could allow a malicious entity to compromise the security of a target machine, and gain access to the data stored on that machine and available on any connected network devices.  Work started in \cite{joshi2013extracting} discusses the need to aggregate data about vulnerabilities from multiple sources to present a more complete and coherent picture, which is also likely to then contain more accurate data.

\cite{Al-zarouni_tracinge-mail} presents an alternative set of results, describing how an individual can seek to protect themselves against malicious e-mails, using the contents of e-mail headers.  Various discrepancies between forged e-mail addressses and legitimate messages are described.