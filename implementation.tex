\chapter{Implementation}\label{chap:imp}

\section{Overview}

In order to satisfy the aims of this project, and building on the presented
objectives, this chapter discusses the implementation of the software intended
to analyse e-mails, starting with the required definitions before presenting
the algorithms and data structures that are needed.

\section{Definitions}

The following covers the essential definitions required for the notation and
concepts that will be discussed in this document.

\subsection{Parsing}

In order to aid the parsing of the e-mail header, a combination of regular
expressions and context-free grammars are needed, and defined as follows.

\paragraph{Alphabets and Languages}

A set of symbols, usually denoted as $\Sigma$.  A language is a subset of
$\mathcal P (\Sigma)$.

The following special classes are provided as part of the Perl-Compatible
Regular Expression library, and are subsets of the alphabet of Unicode
characters, defined in~\cite{php_group_gutmans_lerdorf_suraski_boerger}.
\begin{multicols}{2}
\begin{description}

\item[alnum] --- letters and digits

\item[alpha] --- letters

\item[ascii] --- the set of ASCII characters (character codes 0 --- 127)

\item[blank] --- tabs or blank spaces

\item[cntrl] --- control characters

\item[digit] --- decimal digits

\item[graph] --- printing characters (excluding spaces)

\item[lower] --- lower-case letters

\item[print] --- printing characters (including spaces)

\item[punct] --- punctuation marks (printing characters excluding letters and spaces)

\item[space] --- white space

\item[upper] --- upper case letters

\item[word] --- ``word'' characters

\item[xdigit] --- hexadecimal digits

\end{description}
\end{multicols}
\subsubsection{Regular Languages}
Regular languages are defined as follows:
\begin{multicols}{2}
\begin{itemize}
\item $\emptyset$ and $\{\epsilon\}$ are regular languages
\item for each $a\in\Sigma$, $\{a\}$ is a regular language
\item if $A$ and $B$ are both regular, $A\cup B$, $A\cdot B$ and $A^*$ are regular languages.
	\begin{itemize}
\item $A\cup B$ is the union of two languages.  $A\cup B = \{s : s\in A \lor s \in B\}$
\item $A\cdot B$ is the concatenation of two languages.  $A\cdot B = \{ ab : a \in A, b \in B\}$
\item $A^*$ is the Kleene star of a language.
\begin{align*}
    A_0&=\{\epsilon\}\\
    A_1&= A\\
    A_{i+1} &= \{ aa' : a \in A_i, a'\in A\}\\
    A^* &= \bigcup_{i\in\mathbb N} A_i
\end{align*}
	\end{itemize}
\end{itemize}
\end{multicols}

\subsubsection{Context-Free Grammars}
A context-free grammar $G$ is defined as $G=\left(V,\Sigma, R,S\right)$ where:
\begin{multicols}{2}
\begin{itemize}
\item $V$ is the set of variables or productions.
\item $\Sigma$ is the alphabet of symbols.
\item $R$ is a relation defined over $V\rightarrow \left(V\cup\Sigma\right)^*$
\item $S$ is the start symbol
\end{itemize}
\end{multicols}

For example, $\langle \text S \rangle$ is the field name with the associated
productions $\langle \text T \rangle \, \langle \text U \rangle$, where $T$ and
$U$ are productions.

\begin{bnf*}
	\bnfprod{S}{\bnfpn{T} \bnfsp{} \bnfpn{U}}
\end{bnf*}

For example, $\langle \text S \rangle$ is the field name with the associated
productions $a \, \langle \text U \rangle$, where $a$ is a terminal symbol.

\begin{bnf*}
	\bnfprod{S}{\bnftd{a} \bnfsp{} \bnfpn{U}}
\end{bnf*}
This is then extended in the following ways used in the RFC syntax.

The square brackets are used to indicate an optional element.
\begin{bnf*}
\bnfprod{field}{\bnfpn{field-name} \bnfts{:} \bnfsp{}[ \bnfpn{field-body} ]\bnfsp{} \bnfts{CRLF}}\\
\end{bnf*}

The asterisk is used to indicate an element that appears 0 or more times. $n*$
is used to indicate a component that repeats $n$ or more times.

\begin{bnf*}
\bnfprod{fields}{\bnfpn{dates}\bnfsp \bnfpn{source} \bnfsp 1\!*\bnfpn{destination} \bnfsp * \bnfpn{optional-fields}}\\
\end{bnf*}

The hash-symbol is used to indicate an element that appears a certain number of
times. $m*n$ is used to indicate a component that repeats at least $m$ times and
at most $n$ times.

\begin{bnf*}
\bnfprod{fields}{\bnfpn{dates}\bnfsp \bnfpn{source} \bnfsp 1\!\#\bnfpn{destination} \bnfsp * \bnfpn{optional-fields}}\\
\end{bnf*}

The $|$ is used to indicate a selection between a pair of elements.
\begin{bnf*}
\bnfprod{fields}{\bnfts{a}\bnfor \bnfts{b}}\\
\end{bnf*}

\subsection{Database Queries}
The following notations will be used for the CVE database queries.

\paragraph{Set-Theoretic Operators}
The operators $f\in F, F\cup G$, $F \cap G$, $F\setminus G$ behave as is expected for
these operators, resulting in the union, intersection and difference of the
sets.  The only proviso being that the attribute names must match.

\paragraph{Selection} \[\sigma_{\text{product}=\text{thunderbird}}D\]
The above notation is used to indicate a search over the attribute named
``product'' for the string ``thunderbird'' in the database table $D$.  As a
single database is only being used, this may be occasionally elided.  The output
of this function is another object of the same type as $D$.

\paragraph{Projection}
\[\pi_{\text{product}}D\]
The above notation is used to indicate a projection on the attribute named
``product'' in the database table $D$.  The output of this function is another
object of the same type as $D$.

\paragraph{Composition}
The above functions results can be composed repeatedly to produce more specific
search queries.

\subsection{Data Structures}
These will be drawn using square boxes to represent single objects that are encapsulated
within an object.  Square boxes with an inner square box indicate some collection
of objects.

Thin arrows will be used to denote the encapsulation relation, with thicker
arrows being used to list relevant public methods.

Where the type of an object can be expressed simply, the following conventions are used:
\begin{description}
	\item [Functions] --- $\alpha \rightarrow \beta \, \texttt{Str}$ is a function from $\alpha$ to $\beta$ stored using a \texttt{Str} object in Java.
	\item [Tuples] --- $(\alpha, \beta)$ represents a two-tuple containing objects of type $\alpha$ and $\beta$.  This extends naturally for arbitrarily many objects.
	\item [Lists] --- $[\alpha]$ represents a list or array of objects, all with type $\alpha$.
\end{description}

\section{Data Extraction and Parsing}

The parser's operation completes in a number of stages, following RFC822
(\cite{RFC0822}).  The header is divided up into two disjoint sections, the
routing information (\texttt{Received from...}) and the key-value map of other
pertinent information.

\subsection{Received fields}\label{sec:par}

The received fields are the most complicated part of the e-mail header to parse,
as they are described by a non-trivial grammar, presented below.

\begin{bnf*}
\bnfprod{message}{\bnfpn{fields}\bnfsp *(\bnfts{CRLF} \bnfsp *\bnftd{text})}\\
\bnfprod{fields}{\bnfpn{dates}\bnfsp \bnfpn{source} \bnfsp 1\!*\bnfpn{destination} \bnfsp * \bnfpn{optional-fields}}\\
\bnfprod{field}{\bnfpn{field-name} \bnfts{:} \bnfsp [ \bnfpn{field-body} ]\bnfsp \bnfts{CRLF}}\\
\bnfprod{field-name}{\bnftd{any word consisting of CHAR, excluding CTLs, SPACE, and ``'':''}} \\
\bnfprod{field-body}{\bnfpn{field-body-contents} \bnfsp [\bnfts{CRLF} \bnfsp \bnftd{LWSP-char}\bnfsp  \bnfpn{field-body}]}\\
\bnfprod{field-body-contents}{\bnftd{ASCII characters}}\\
\bnfprod{source}{[\bnfpn{trace}] \bnfsp \bnfpn{originator} [\bnfpn{resent}]}\\
\bnfprod{trace}{\bnfpn{return}\bnfsp 1\!* \bnfpn{received}}\\
\bnfprod{return}{\bnfts{Return-path:}\bnfsp{} \bnfpn{route-addr}}\\
\bnfprod{received}{\bnfts{Received:}}\\
\bnfprod{cont.}{[\bnfts{from}\bnfsp\bnfpn{domain}]}\\
\bnfprod{cont.}{[\bnfts{by}\bnfsp\bnfpn{domain}]}\\
\bnfprod{cont.}{[\bnfts{via}\bnfsp\bnfpn{atom}]}\\
\bnfprod{cont.}{*(\bnfts{with}\bnfsp\bnfpn{atom})}\\
\bnfprod{cont.}{[\bnfts{id}\bnfsp\bnfpn{msg-id}]}\\
\bnfprod{cont.}{[\bnfts{for}\bnfsp\bnfpn{addr-spec}]}\\
\bnfprod{cont.}{\bnfts{;}\bnfsp\bnfpn{date-time}}\\
\bnfprod{msg-id}{\bnfts{$<$}\bnfpn{addr-spec}\bnfts{$>$}}\\
\bnfprod{addr-spec}{\bnfpn{local-part}\bnfsp\bnfts{@}\bnfsp\bnfpn{domain}}\\
\bnfprod{local-part}{\bnfpn{word}\bnfsp *(\bnfts{.}\bnfsp\bnfpn{word})}\\
\bnfprod{word}{\bnfpn{atom}\bnfor\bnfpn{quoted-string}}\\
\bnfprod{domain}{\bnfpn{sub-domain} *(\bnfts{.}\bnfpn{sub-domain})}\\
\bnfprod{sub-domain}{\bnfpn{domain-ref}\bnfor\bnfpn{domain-literal}}\\
\bnfprod{domain-ref}{\bnfpn{atom}}\\
\bnfprod{date-time}{[ \bnftd{day,} ] \bnfsp \bnftd{date}\bnfsp \bnftd{time}}\\
\bnfprod{atom}{1\!*\bnftd{any character excluding specials, SPACE and CTLs}}\\
\end{bnf*}
\clearpage
An example field is as follows:
\begin{verbatim}
Received: from relay12.mail.ox.ac.uk (129.67.1.163)
    by HUB05.ad.oak.ox.ac.uk (163.1.154.231)
    with Microsoft SMTP Server id 14.3.169.1;
    Sat, 14 Nov 2015 10:55:35 +0000
\end{verbatim}

Of particular interest is the pair of hostnames (both are needed as each line
is analysed independently, therefore it is not necessary to treat the last line
as a special case) and their associated IP addresses.  The hostname is of more
interest, as performing an IP address lookup is less complicated than
determining a hostname.  After finding the IP address, it is then often
possible to perform a GeoIP lookup to acquire an approximate location for the
device. Sometimes, by determining the hostname, it is then possible to perform
a WhoIs search for information about the owner of the domain. The software used
is identified by the \texttt{with} field, and is also of interest, however, is
insufficient in most cases to produce meaningful CVE data.

\subsection{Other fields}

These are read by a Python script and output to \texttt{STDOUT} to be read by
the Java parser in a consistent format.  These are then loaded into a hash-map to
allow quick lookup.


\subsection{Input Data Structures}
The raw string of the message header is the only input to this module.

\subsection{Output Data Structures}
The data structure presented in Figure~\ref{fig:hea} shows the output of the
parsing and textual analysis module, which is then provided as an input to the
analysis modules.

\begin{figure}
\centering
\resizebox{0.9\textwidth}{!}{
	\begin{tikzpicture}
		\draw (0,0) rectangle node {Header} (4,2);
		\draw (0,3) rectangle node [text width = 3.5cm, align=center] {Start Device\\Type: Device} (4,5);
		\draw (5,0) rectangle node [text width = 3.5cm, align=center]
		{Other fields\\String $\rightarrow$ String \\ \texttt{HashMap}} (9,2);
		\draw [-{Latex[length=3mm]}] (2,2) -- (2,3);
		\draw [-{Latex[length=3mm]}] (4,1) -- (5,1);
		\draw [-{Latex[length=3mm]}] (4,4) -- (5,4) node [right, text width = 4cm,align=center] { Name, Software, Latitude, Longitude, Next Device (of type Device), Received Time, Owner};
		\draw [->, ultra thick] (0,1.5) -- (-1,1.5) node [left, text width = 2.5cm,align=center] { Getter/Setter for Device};
		\draw [->, ultra thick] (0,0.5) -- (-1,0.5) node [left, text width = 2.5cm, align=center] { Getter/Setter for K/V Map};
\end{tikzpicture}
	}
	\caption{Header Data Structure Format}
	\label{fig:hea}
\end{figure}


\section{Analysis}

After completing the parsing of the fields, it is then ready to be analysed for
different features.  All of the analysers implement the \texttt{HeaderAnalyser}
interface, requiring information about the header to be analysed, and the
currently running application.  All of these then implement the
\texttt{Runnable} interface, allowing the class to be run asynchronously.

\subsection{Text-Based}

The fields from the header are analysed in different modules, with searches
being performed for specific strings.  Of particular interest to Oxford Nexus
users is the ``X-Oxford-Username'' string, containing the username of the
individual that sent the message.  As confirming the username is a fairly
standard security procedure for an IT support technician, having access to this
information could allow a phisher in a later stage of an attack to increase
their credibility.

In some cases, the likely keys that are being searched for are known in
advance, and can then be checked against the hash-map of entries.

An example of this approach is for the specific check for an Oxford username,
as shown in Algorithm~\ref{alg:oxf}.

\begin{algorithm}
	\KwIn{Header}
	\KwOut{Any username that is found}
$kws \gets \{\texttt{X-Oxford-Username}, \texttt{X-Username}, \texttt{X-Authenticated-User}\}$\;
\ForEach{$kw\in kws$}{
	\If{$kw\in$ Header.KvMap}{ \Return{Header.KvMap(kw)}\; }
}
	\caption{Lookup based on a known key}
	\label{alg:oxf}
\end{algorithm}

Alternatively, we may be interested in properties of the keys, necessitating a search over the keys, as shown in Algorithm~\ref{alg:exc}.

\begin{algorithm}
	\KwIn{Header}
	\KwOut{Any information relating to Microsoft Exchange that is found}
	\ForEach{Key $k\in $ Header.KvMap}{
		\If{$k$ starts-with \texttt{X-MS-Exchange}}{
			\Return{Header.KvMap($k$)}\;
		}
	}
	\caption{Lookup based on a key property}
	\label{alg:exc}
\end{algorithm}

\subsection{Client Inference}

In \cite{nurse2015investigating}, a number of different e-mail clients were
identified based on the header tags that were present. By determining these
pieces of software likely to be found on a user's machine, we gain a
significant amount of information from them, and should therefore devote some
effort to correctly identifying them.  Using a number of e-mail samples
provided, I have been able to extract examples for a number of different e-mail
clients, and use them to infer the client being used.  A number of these
approaches are shown in Algorithm~\ref{alg:inf}.

\begin{algorithm}
	\KwIn{Header}
	\KwOut{The name of the software that is likely being used, and necessary CVE Product name (elided for brevity)}
    OutlookKeywords ${}\gets\{ \text{Accept-Language}, \text{Content-Language}, \text{Threat-Index}, \text{Threat-Topic} \}$\;
	\uIf{``Message-ID''${}\in{}$Header.KvMap}{
		\If{Header.KvMap(``Message-ID'') contains ``email.android.com''}{
			\Return{``Android Device''}\;
		}
	}
	\uElseIf{``X-Mailer''${}\in{}$Header.KvMap}{
		\Switch{Header.KvMap(``X-Mailer'') contains}{
			\Case{``iPhone''}{\Return{``iPhone''}}
			\Case{``Outlook Express''}{\Return{``Microsoft Outlook Express''}}
			\ldots
		}
	}
	\uElseIf{``User-Agent''${}\in{}$Header.KvMap}{
		\Return{``Thunderbird''}\;

	}
	\ElseIf{Header.KvMap${}\cap{}$OutlookKeywords${}\neq\emptyset$}{
		\eIf{``X-Mailer''${}\in{}$Header.KvMap}{ \Return{Apple Mail}}{ \Return{Outlook}}

	}
	\caption{Client Inference Technique}
	\label{alg:inf}
\end{algorithm}


\subsection{Database Queries}

Using the results gathered from the text-based queries and analysis of the
received fields, relevant software configurations are extracted and queried
against results in the CVE database.  These are then parsed and collated in
preparation for displaying the outputs. Specifically, the queries are limited
to those matching the product name, and vulnerabilities that can be remotely
executed.  This process is outlined in Algorithm~\ref{alg:db}.

As more information is found, more details of products used will also become
available.  These are added asynchronously.

\begin{algorithm}
	\KwIn{Header product name $p$}
	\KwOut{CVE Entries}
	cve-list $\gets\emptyset$\;
	\ForEach{$s\in\sigma_{\text{vector}\neq\text{LOCAL}}\sigma_{\text{product}=p} D$}{
		cve-builder$\gets$blank cve\;
		cve-builder.id$\gets\pi_{\text{CVE-ID}} s$\;
		$\ldots$ -- extract other features\;
		cve-list $\gets$ cve-list${}\cup{}$make(cve-builder)\;
	}
	\Return{cve-list}\;
	\caption{Extracting CVE entries}
    \label{alg:db}
\end{algorithm}

\subsection{Analysis Modules and Data Flow}

The following modules are used in the analysis of e-mail headers. Via
\texttt{HeaderAnalyser}, they all subclass \texttt{Callable<Object>}, allowing
them to return values to calling classes when side-effects are undesirable.

\begin{description}
	\item[HeaderAnalyser] --- the base interface for all analysis modules.

	\item[ClientInferrer] --- as described in Algorithm~\ref{alg:inf}, this
		analyses the entire header, looking for specific indicators
		relating to e-mail clients.
		
	\item[DeviceAnalyser] --- Extracts the hostname, and then IP address, or
		IP address for each device, allowing a lookup of its
		co-ordinates.

	\item[ExchangeHeaderAnalyser] --- determines if a Microsoft Exchange
		server has handled the message.

	\item[GeoIPAnalyser] --- Given an IP address, this module looks up the
		latitude and longitude for said IP address.  This search will
		fail for local IP addresses (commonly found in the
		192.168.0.0/16 subnet).

	\item[UsernameHeaderAnalyser] --- This module searches for the specific
		\texttt{X-Oxford-Username} field, one of the most common fields
		in my collection of e-mail headers, as well as a number of more
		generic username fields.

	\item[SenderInformationExtractor] --- This is used to lookup an
		individual's name from the set of fields, if it is available.

	\item[VulnerabilityAnalyser] --- For each entry from the database as a
		\texttt{String}, this analyser returns a
		\texttt{VulnerabilityDisclosure} object.

	\item[VulnerabilityFinderManager] --- this is an interface for other
		classes to implement. A reference implementation is provided in
		\texttt{VulnerabilityFinderManagerImpl}, with a simple
		implementation found in \texttt{NoopVulnerabilityFinderManager}
		for systems that lack the CVE database.

	\item[WhoIsAnalyser] --- Using a hostname, this looks up the relevant
		owning organisation for a server, if it is available.
		
\end{description}

Figure~\ref{fig:flo} shows the direct interaction of the analysis modules.
Unless noted, if an arrow goes from $\alpha$ to $\beta$, it is used to indicate
that $\alpha$ calls $\beta$, passing data from $\alpha$, and/or receiving data
from $\beta$.

Where possible, these requests are non-blocking, that is run asynchronously,
while waiting for other results to be computed.  The biggest delay is caused by
the CVE Database lookup, followed by the WhoIs lookup and then the GeoIP lookup.
The CVE database lookups require the most time as they are often
consuming large amounts of data, and searching over specific fields.  The WhoIs
lookup takes place over the network, causing its response time to be
unpredictable, however, its relatively small response size reduces the time
needed.  Finally, the GeoIP lookup takes place locally, with a relatively small
response, allowing it to complete quickly.

\begin{figure}
	\resizebox{0.9\textwidth}{!}{
\begin{tikzpicture}
\tikzset{every node/.style={align=center, text width = 3.5cm, font = \ttfamily}}
\draw (0,0) rectangle node {Main Window} (4,2);
\draw [-{Latex[length=3mm]}] (4,1) -- (5,1);
\draw (5,0) rectangle node {Username Analyser} (9,2);
\draw [-{Latex[length=3mm]}] (4,2) -- (5,3);
\draw (5,3) rectangle node {Client Inferrer} (9,5);
\draw (0,-1) rectangle node {Exchange Header Analyser} (4,-3);
\draw [-{Latex[length=3mm]}] (2,0) -- (2,-1);
\draw (0,3) rectangle node {Device Analyser} (4,5);
\draw [-{Latex[length=3mm]}] (2,2) -- (2,3);
\draw (5,6) rectangle node {Vulnerability Finder Manager} (9,8);
\draw [-{Latex[length=3mm]}] (4,5) -- (5,6);
\draw [-{Latex[length=3mm]}] (7,5) -- (7,6);
\draw (10,6) rectangle node {Vulnerability Analyser} (14,8);
\draw [-{Latex[length=3mm]}] (9,7)  -- (10,7);
\draw (0, 6) rectangle node {Geo-IP Analyser} (4,8);
\draw [-{Latex[length=3mm]}] (2,5) -- (2,6);
\draw (-5,0) rectangle  node {Sender Information Analyser} (-1,2);
\draw [-{Latex[length=3mm]}] (0,1)  -- (-1,1);
\draw (-1,6) rectangle node {Who-Is Analyser} (-5,8);
\draw [-{Latex[length=3mm]}] (0,5) -- (-1,6);
\draw [-{Latex[length=3mm]}] (-3,2) -- (-3,6);
\draw [-{Latex[length=3mm]}] (-3,8) -- (-3,9);
\draw [-{Latex[length=3mm]}] (2,8) -- (2,9);
\draw [-{Latex[length=3mm]}] (7,8) -- (7,9);
\draw (-1,9) [dashed] rectangle node {External Lookup} (-5,11);
\draw (0,9) [dashed] rectangle node {External Lookup} (4,11);
\draw (5,9) [dashed] rectangle node {External Lookup} (9,11);
\end{tikzpicture}}
\caption{Information flow between analysis modules}
\label{fig:flo}
\end{figure}

\subsection{Output Data Structures}

The output of the analysis modules is compiled into the
\texttt{FoundInformation} class, which represents the list of facts, servers
and other information that has been gathered from an e-mail.

\begin{figure}
	\centering
\resizebox{0.9\textwidth}{!}{%
\begin{tikzpicture}
\draw (0,0) rectangle node {Found Information} (4,2);
\draw [->, ultra thick] (0,0.5) -- (-1,-0.5) node [left, text width = 3cm, align=center] {Add Username (context, data) };
\draw [->, ultra thick] (0,1) -- (-1,1) node [left, text width = 3cm, align=center] {Add Fact (class, context, data) };
\draw [->, ultra thick] (0,1.5) -- (-1,2.5) node [left, text width = 3cm, align=center] {Add Device (IP, location, software) };
\draw [-{Latex[length=3mm]}] (2,2) -- (2,3);
\draw [fill=white] (-0.25, 3.25) rectangle  (-0.25+4, 3.25+2);
\draw [fill=white] (-0, 3) rectangle node [text width = 3.5cm, align=center] {Fact List\\(Type: [(String, String, String)])} (-0+4, 3+2) ;
\draw [fill=white] (-.25+5, 3.25) rectangle (-0.25+5+4, 3.25+2);
\draw [fill=white] (5, 3) rectangle node [text width = 3.5cm, align=center] {User ID List (Type: \\{} [(String, String)])} (5+4, 3+2);
\draw [-{Latex[length=3mm]}] (4,2) -> (5,3);
\draw [fill=white] (-0.25+5, 0.25) rectangle (8.75, 2.25);
\draw [fill=white] (5, 0) rectangle node [text width = 3.5cm, align = center] {Device List\\ Type: [DeviceInfo]} (9, 2);
\draw [-{Latex[length=3mm]}] (4,1) -> (4.75,1);
\draw [->, ultra thick] (2,0) -- (2,-1) node [below, text width = 4.75cm, align = center] {Getters for name, software, User ID, device, fact lists};
\draw [-{Latex[length=3mm]}] (4,0) -> (5,-1);
\draw (5,-1) rectangle node [align=center] {User name,\\ software} (9,-3);
\draw (1+10,0-1.5) rectangle node {Device Info} (1+14,2-1.5);
\draw (1+10,3-1.5) rectangle node [align=center, text width = 3.5cm]{ID, IP Address, Latitude, Longitude, Software, Organisation} (1+14,5-1.5);
\draw [-{Latex[length=3mm]}] (1+12,2-1.5) -- (1+12,3-1.5);
\end{tikzpicture}}
\caption{Found Information Data Structure Format}
\label{fig:fou}
\end{figure}

\section{Visualising the Results}

Using a pre-existing template, the results from the e-mail analysis will be
presented in a temporary webpage, which can then be saved independently.  Other
than the referenced JavaScript libraries, and a freely available set of country 
borders, the document requires no additional information or database access, 
allowing it to be quickly shared.

In order to export the results, the process described in
Algorithm~\ref{alg:vis} are taken.  The \textit{keyword} objects referred to are
one of a number of specific strings representing one of the sender's name,
organisation, client or usernames; CVE entries, fact entries, products that
have been found, or servers that have been found.

\begin{algorithm}
	\KwIn{Found Information Object}
	\KwOut{Webpage}
	$\textit{lines}\gets\text{read-lines(webpage-template)}$\;
	\ForEach{$l\in\text{lines}$}{
		\eIf{$l$ contains keyword}{
		$\textit{webpage}\gets_+ l[\text{field from FoundInformation}/\text{keyword}]$\;
		}{
		$\textit{webpage}\gets_+ l$\;
		}
	}
	\Return{webpage}\;
	\caption{Exporting the Found Information to Visualisations}
	\label{alg:vis}
\end{algorithm}

While most of the data processing has been completed by the time the webpage is 
displayed, some computational steps are required to render the histograms and
map, which are detailed in Algorithm~\ref{alg:vjs}

\begin{algorithm}
	\KwIn{Webpage text}
	\KwOut{Visualisation of Webpage}
	\lForEach{$u\in\text{userentries}$}{find HTML list of usernames and append $u$}
	\lForEach{$f\in\text{factentries}$}{find HTML table of facts and append $f$}
	\ForEach{$c\in\text{cveentries}$}{
		\If{$c\text{ matches search string}\lor\text{search string}\stackrel{?}{=}\epsilon$}{
			\If{$\langle c_\text{impact}, c_\text{access}\rangle$ matches filters}{ 
				find HTML table of facts and append $f$\;
			}
		}
	}
	\lForEach{$c\in\text{cveentries}$}{
		$\text{distributions}\left[c_\text{software}\right]\gets_\cup c_\text{score}$
	}
	\lForEach{$d\in\text{distributions}$}{draw histogram for $d$}
	\ForEach{$s_i\in\text{servers}$}{project $s_i$ onto map\;add $s_i$ to server table\;}
	\lForEach{$s_i\in\text{servers}$}{draw link between $s_i$ and $s_{i+1}$ onto map}

	\Return{webpage}\;
	\caption{Rendering the Visualisation}
	\label{alg:vjs}
\end{algorithm}
