\chapter{Definitions}
\section{Parsing}
In order to aid the parsing of the e-mail header, a combination of regular expressions and context-free grammars are needed, and defined as follows.
\subsection{Alphabets and Languages}
A set of symbols, usually denoted as $\Sigma$.  A language is a subset of $\mathcal P (\Sigma)$.

The following special classes are provided as part of the Perl-Compatible Regular Expression library, and are subsets of the alphabet of Unicode characters, defined in \cite{php_group_gutmans_lerdorf_suraski_boerger}.
\begin{description}
\item[alnum] -- letters and digits
\item[alpha] -- letters
\item[ascii] -- the set of ASCII characters (character codes 0 --- 127)
\item[blank] -- tabs or blank spaces
\item[cntrl] -- control characters
\item[digit] -- decimal digits
\item[graph] -- printing characters (excluding spaces)
\item[lower] -- lower-case letters
\item[print] -- printing characters (including spaces)
\item[punct] -- punctuation marks (printing characters excluding letters and spaces)
\item[space] -- white space
\item[upper] -- upper case letters
\item[word] -- ``word'' characters (same 
\item[xdigit] -- hexadecimal digits
\end{description}
\subsection{Regular Languages}
Regular languages are defined as follows:
\begin{itemize}
\item $\emptyset$ and $\{\epsilon\}$ are regular languages
\item for each $a\in\Sigma$, $\{a\}$ is a regular language
\item if $A$ and $B$ are both regular, $A\cup B$, $A\cdot B$ and $A^*$ are regular languages.
\subitem $A\cup B$ is the union of two languages.  $A\cup B = \{s : s\in A \lor s \in B\}$
\subitem $A\cdot B$ is the concatentation of two languages.  $A\cdot B = \{ ab : a \in A, b \in B\}$
\subitem $A^*$ is the Kleene star of a language. 
\begin{align*}
    A_0&=\{\epsilon\}\\
    A_1&= A\\
    A_{i+1} &= \{ aa' : a \in A_i, a'\in A\}\\
    A^* &= \bigcup_{i\in\mathbb N} A_i
\end{align*}
\end{itemize}

\subsection{Context-Free Grammars}
A context-free grammar $G$ is defined as $G=\left(V,\Sigma, R,S\right)$ where:
\begin{itemize}
\item $V$ is a variable.
\item $\Sigma$ is the alphabet of symbols.
\item $R$ is a relation defined over $V\rightarrow \left(V\cup\Sigma\right)^*$
\item $S$ is the start symbol
\end{itemize}

For example, $\langle \text S \rangle$ is the field name with the associated productions $\langle \text T \rangle \, \langle \text U \rangle$, where $T$ and $U$ are productions.
\begin{bnf*}
	\bnfprod{S}{\bnfpn{T} \bnfsp \bnfpn{U}}
\end{bnf*}
For example, $\langle \text S \rangle$ is the field name with the associated productions $a \, \langle \text U \rangle$, where $a$ is a terminal symbol.
\begin{bnf*}
	\bnfprod{S}{\bnftd{a} \bnfsp \bnfpn{U}}
\end{bnf*}
This is then extended in the following ways used in the RFC syntax.

The square brackets are used to indicate an optional element.
\begin{bnf*}
\bnfprod{field}{\bnfpn{field-name} \bnfts{:} \bnfsp [ \bnfpn{field-body} ]\bnfsp \bnfts{CRLF}}\\
\end{bnf*}

The asterisk is used to indicate an element that appears 0 or more times. $n*$ is used to indicate a component that repeats $n$ or more times.
\begin{bnf*}
\bnfprod{fields}{\bnfpn{dates}\bnfsp \bnfpn{source} \bnfsp 1\!*\bnfpn{destination} \bnfsp * \bnfpn{optional-fields}}\\
\end{bnf*}

The hash-symbol is used to indicate an element that appears a certain number of times. $m*n$ is used to indicate a component that repeats at least $m$ times and at most $n$ times.
\begin{bnf*}
\bnfprod{fields}{\bnfpn{dates}\bnfsp \bnfpn{source} \bnfsp 1\!\#\bnfpn{destination} \bnfsp * \bnfpn{optional-fields}}\\
\end{bnf*}

The $|$ is used to indicate a selection between a pair of elements.
\begin{bnf*}
\bnfprod{fields}{\bnfts{a}\bnfor \bnfts{b}}\\
\end{bnf*}
