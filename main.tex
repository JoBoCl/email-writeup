\documentclass{article}
\usepackage[utf8]{inputenc}

\title{Introduction Writeup}
\author{Joshua Clark}
\date{October 2015}

\usepackage{natbib}
\usepackage{graphicx}

\begin{document}

\maketitle

\section{Motivation}
After extensive public education, fewer people are now clicking on links in e-mails that are disguised as phishing attacks, though the threat still remains, and considerable amounts of work has gone into exploring the demographics most likely to be targeted.  As the number of technically literate people grows, this sort of attack is increasingly unlikely to be successful.  Therefore, malicious entities are more likely to attempt to attack people based on the information leaked in their emails, and more specifically, the header, which most people are less likely to have some degree of control over.  
\section{Research}
In Nurse's seminal paper \cite{nurse2015investigating}, the idea of using the information available in an email header was mooted, turning the previously standard threat of malware and phishing contained in received e-mails on its head, and instead presenting the threat in outgoing emails, and the personally identifying information (PII) contained therein.  Many emails leaked information about employers, e-mail services and applications used, and IP address.  Initial examination of a variety of e-mail headers found within my own inbox also revealed a plethora of information, including phone carriers, preferred languages, and system usernames.  It is conceivable therefore, that it is possible to automate at least part of this, and present the information that can be extracted, in a white-hat tool to allow people to audit the information that they are revealing.  The obvious malicious use-case involves using such information as part of a spear-phishing exercise.

An alternative vulnerability presents itself in the information about systems that may be revealed.  Many email clients embed identifying information, and there are multiple databases available to allow specific threats to be identified.  This could allow a malicious entity to compromise the security of a target machine, and gain access to the data stored on that machine and available on any connected network devices.  Work started in \cite{joshi2013extracting} discusses the need to aggregate data about vulnerabilities from multiple sources to present a more complete and coherent picture, which is also likely to then contain more accurate data.
\section{Project Aims}
Given the work already completed, I am looking to present a tool to automatically analyse e-mail headers and present a list of social and technical vulnerabilities.  It should have the following features:
\begin{itemize}
\item Identify a graph of mail servers and devices that the message has passed through
\item Present any personally identifying information to the user found from the given header
\item List vulnerabilities found online for specific versions of software.
\item Produce a final evaluation for an email header indicating the amount of PII that has been found, as well as any vulnerabilities and the threat that they would pose to a system.

This should either be a numerical value, or something based on priority levels.
\item \emph{Bonus} -- Given a set of e-mails sent by the same person, use further emails to expand the information available, and verify the existing information
\end{itemize}
\section{Timetable}
\begin{description}
\item[Michaelmas] Parse e-mails, create lookup for frequently appearing fields
\item[Christmas Vac] Preliminary work on fetching CVE entries
\item[Hilary] Complete CVE work
\item[Easter Vac] Present information in UI
\item[Trinity] Final fixes and evaluation writeup
\end{description}
At all stages, the work that has been completed should be documented in the project write-up.
\bibliographystyle{plain}
\bibliography{references}
\end{document}
