\chapter{Conclusions and Future Work}\label{chap:conc}
\section{Conclusions}

\section{Future Work}

During the late stages of development and testing, a number of missing features
quickly became apparent. Due to the limited information available, and the
differences in version numbering, a decision was made to search for all
available vulnerabilities for an application, allowing the user to discern
which were most relevant.  Subsequent versions could focus on the different
pieces of version data available.  For example, \texttt{esmtp} frequently
references its version number in the ``Received'' field frequently.

Alternatively, a better picture may be presented by accumulating multiple
e-mails. For example, using the information provided from multiple members of
single organisation, a better picture may be built up of the software used by
the servers, as well as the network configuration.

Additionally, future testing should take place on a larger dataset, using
e-mails from a wider variety of sources sent to a number of different
recipients.  

Finally, it should be possible for an updated version of this application to
determine which header fields have been added by mail servers within one's own
organisation.  For example, e-mail header fields beginning with
\texttt{X-MS-Exchange} are seen within almost all e-mail messages sent to
recipients within the Oxford domain, adding more false positives to the test
results. While it is possible that other preceding e-mail servers have added
similar fields, in most cases, these entries yielded little useful data.

